% !TXS template
\documentclass[french]{article}
\usepackage[T1]{fontenc}
\usepackage[utf8]{inputenc}
\usepackage{lmodern}
\usepackage[a4paper]{geometry}
\usepackage{babel}
\usepackage{listings}
\usepackage{xcolor}
\usepackage{amsmath}
\usepackage{amsfonts}
\usepackage{graphicx}

\definecolor{mygreen}{rgb}{0,0.6,0}
\definecolor{mygray}{rgb}{0.9,0.9,0.9}
\definecolor{mymauve}{rgb}{0.58,0,0.82}

\lstset{
    literate= {≡}{{$\equiv$}}1,
    linewidth=12cm,
    xleftmargin=.1\textwidth,
    backgroundcolor=\color{mygray},   % choose the background color; you must add \usepackage{color} or \usepackage{xcolor}; should come as last argument
    basicstyle=\tiny\ttfamily,        % the size of the fonts that are used for the code
    breakatwhitespace=false,         % sets if automatic breaks should only happen at whitespace
    breaklines=true,                 % sets automatic line breaking
    captionpos=b,                    % sets the caption-position to bottom
    commentstyle=\color{mygreen},    % comment style
    deletekeywords={...},            % if you want to delete keywords from the given language
    escapeinside={\%*}{*)},          % if you want to add LaTeX within your code
    extendedchars=true,              % lets you use non-ASCII characters; for 8-bits encodings only, does not work with UTF-8
    firstnumber= 0,                % start line enumeration with line 0
    frame=single,	                   % adds a frame around the code
    keepspaces=true,                 % keeps spaces in text, useful for keeping indentation of code (possibly needs columns=flexible)
    keywordstyle=\color{blue},       % keyword style
    language=Python,                 % the language of the code
    morekeywords={*,...},            % if you want to add more keywords to the set
    numbers=left,                    % where to put the line-numbers; possible values are (none, left, right)
    numbersep=5pt,                   % how far the line-numbers are from the code
    numberstyle=\tiny, % the style that is used for the line-numbers
    rulecolor=\color{black},         % if not set, the frame-color may be changed on line-breaks within not-black text (e.g. comments (green here))
    showspaces=false,                % show spaces everywhere adding particular underscores; it overrides 'showstringspaces'
    showstringspaces=false,          % underline spaces within strings only
    showtabs=false,                  % show tabs within strings adding particular underscores
    stepnumber=1,                    % the step between two line-numbers. If it's 1, each line will be numbered
    stringstyle=\color{mymauve},     % string literal style
    tabsize=2,	                   % sets default tabsize to 2 spaces
    title=\lstname                   % show the filename of files included with \lstinputlisting; also try caption instead of title
}
\author{
    DE MATOS Kevin\\
    \and
    LEWANDOWSKI Basile\\
}
\title{COMPLEX - Projet}
\begin{document}
\maketitle
\section*{Exercice 1}
\paragraph{a.} Calcul du pgcd selon l'algorithme d'Euclide :
Pour chaque itération, on effectue "a mod b"\\
 - Si le résultat est différent de 0, "a" prend la valeur "b" et "b" la valeur "a mod b"\\
 - Si le résultat est égal à 0, on retourne la valeur de "b" correspondant au pgcd(a,b)
 \begin{lstlisting}[language=Python, belowskip=-1 \baselineskip]
def my_gcd(a,b): 
     while(a%b != 0) : 
        a, b = b, a%b 
     return b
\end{lstlisting}
\paragraph{b.}Calcul de l’inverse du modulo N de a selon l'algorithme d’Euclide étendu:\\
Utilisation de l'identité de Bézout avec ab + Nc = PGCD(a, N)\\
Initialisation avec une matrice identité pour les valeurs de b,b1,c,c1\\
Retourne un entier b tel que a*b $\equiv$ 1 mod N ssi pgcd(a,N) = 1\\
\begin{lstlisting}[language=Python]
def my_inverse(a,N):
    b,b1 = 1,0
    c,c1 = 0,1
    while N!=0:
        temp_quotient, temp_reste = a//N, a%N 
        a, N = N, temp_reste
        b, b1 = b1, (b - temp_quotient*b1)
        c, c1 = c1, (c - temp_quotient*c1)

    # a => pgcd(a,N)
    # b => a*b ≡ pgcd(a,N) mod N
    # c => N*c ≡ pgcd(a,N) mod a

    if a!=1:
        raise ValueError("a n'est pas inversible modulo N!")
    else:
        return b
\end{lstlisting} 
\newpage 
\paragraph{c.} \*\\
\begin{lstlisting}[language=Python, belowskip=-1 \baselineskip]
    import time
    import matplotlib.pyplot as plt
    
    complexity1 = []
    complexity2 = []
    
    for i in range (1,3000):
        start1=time.time()
        temp1 = my_gcd(89**i,97**i)
        complexity1.append(time.time()-start1)
    
        start2=time.time()
        temp2 = my_inverse(89**i,97**i)
        complexity2.append(time.time()-start2)
    
    plt.plot(complexity1,label="my_gcd()")
    plt.plot(complexity2,label="my_inverse()")
    plt.legend()
    plt.show()
\end{lstlisting}
\begin{center}
\includegraphics[width=.6\textwidth]{complex.png}
\end{center}
\paragraph{d.} \
\begin{lstlisting}[language=Python, belowskip=-1 \baselineskip]
def my_expo_mod(g,n,N):
    l = n.bit_length()
    h = 1
    for i in range(l,-1,-1):
        h = (h*h)%N
        if((n & (2**i)) == 2**i):
            h = (h*g)%N
    return h
\end{lstlisting}
\section*{Exercice 2}
\paragraph{a.}  
\begin{center}
\begin{lstlisting}[language=Python, belowskip=-1 \baselineskip]
    def first_test(n):
        for a in range(2, int(ma.sqrt(n) + 1)):
            if n % a == 0:
                return False
        return True
\end{lstlisting}
\end{center}
\paragraph{b.} En considérant que la divison euclidienne a une complexité arithmétique en $O(1)$, le test naïf de primalité à une complexité en $O(\sqrt{n})$.
\paragraph{c.} On compte ainsi 9680 nombres premiers inférieurs à $10^5$.

\paragraph{d.} \
\begin{lstlisting}[language=Python, belowskip=-1 \baselineskip]
def gen_carmichael(t):
    res = []
    for x in range(3, int(t), 2):  # les nombres de Carmichael sont impairs
        valid = False
        for y in range(2, x):
            if ma.gcd(x, y) == 1:
                if pow(y, x-1, x) != 1:
                    valid = False
                    break
            else:
                valid = True
        if valid:
            res.append(x)
    return res
\end{lstlisting}

\paragraph{e.} \
\begin{lstlisting}[language=Python, belowskip=-1 \baselineskip]
def gen_carmichael_3(k):
    prime = []
    for n in range(3, 2 ** k, 2):
        if first_test(n):
            prime.append(n)
    res = []
    for i_a, a in enumerate(prime):
        for i_b, b in enumerate(prime[:i_a]):
            ab = a * b
                for c in prime[:i_b]:
                # on  a obtenu 3 premiers, on teste si leur produit est Carmichael
                tst = ab * c - 1
                if ma.ceil(ma.log2(tst)) != k:
                    continue
                if tst % 2 == 0 and tst % (a - 1) == 0 and tst % (b - 1) == 0 
                   and tst % (c - 1) == 0 and a % (b * c) != 0:
                    res.append(tst + 1)
    return sorted(res)
\end{lstlisting}
\paragraph{f.} On compare le nombre de Carmichael maximal obtenu par plusieurs algorithmes :\\
\begin{center}
    
    \begin{tabular}{lll}
        Algorithme naïf :&63973 en 287s\\
        Algorithme naïf multithreadé : & 115 921 en 317s\\
        Algorithme basé sur le critère de Korselt : &1 196 508 858 409 en 302s\\
    \end{tabular}
    
\end{center}
L'algorithme \verb*|gen_carmichael_3| est donc significativement plus efficace. Cependant, il ne retourne pas tous les nombres de Carmichael puisqu'il ne s'intéresse qu'à ceux qui se décomposent en 3 facteurs premiers. Dans cet exemple, il y a 8314 nombres inférieurs au maximum (la majorité) qui n'ont pas été pris en compte.
\paragraph{g.}\subparagraph{1} Soit n un nombre de Carmichael de la forme pqr avec p < q < r trois nombres premiers. D'après le critère de Korselt on a :\\
$\exists \:\lambda \in \mathbb{N}$ tel que :
\begin{align*}
    \lambda(r-1)&=n-1\\
    \lambda(r-1)&=pqr-1\\
    \lambda(r-1)&=pq(r-1)+pq-1\\
    (r-1)(\lambda-pq)&=pq-1
\end{align*}
Soit $h := \lambda - pq$ on a :\\
 - $h \ne 1$ car cela impliquerait $pq=r$ et n'aurait un facteur carré (contraire au critère de Korselt).\\
 - Puisque $r-1>q$ on a $hq<pq$ donc $h<p$ et plus précisement $h \leq p-1$.\\
D'où : $h \in [\![2; p-1]\!]$
\subparagraph{2} De la même manière on peut montrer : $\exists \:k \in [\![p+1; r-1]\!], \: k(q-1)=pr-1$
\begin{align*}
    h(r-1)&=pq-1\\h(r-1)&=p(q-1)+p-1\\h(pr -1 +1 -p)&=p(p(q-1)+p-1)\\
    h(k(q-1) +1 -p)&=p^2(q-1)+p(p-1)\\
    hk(q-1)+h(1-p)&=p^2(q-1)p(p-1)\\
    (p^2+hk)(q-1)&=(h+p)(p-1)
\end{align*}
\subparagraph{3} Ainsi, on a montré que l'on pouvait exprimer q-1 en fonction de p, h et k. On peut également observer d'après (1) qu'on a : $r \leq \frac{1}{2}(pq + 1)$, ce qui nous permet de montrer que les diviseurs premiers de n sont bornés proportionellement à p.\\
En effet, soit $p \in \mathbb{N}$, on a : $q - 1 \leq (2p - 1)(p-1)$ donc $q\leq 2(p^2 + 1)$ \\ D'où $r \leq p(p^2 + 1) + \frac{1}{2}$ ce qui montre que r est borné, c'est à dire qu'il n'existe qu'un nombre fini de nombres de
Carmichael de la forme $p*q*r$ pour p fixé.
\paragraph{h.}  \*\\ Pour p = 3 on obtient : 561 \\ Pour p = 5 on obtient : 1105, 2465, 10585
\section*{Exercice 3}
\paragraph{a.} \
\begin{lstlisting}[language=Python, belowskip=-1 \baselineskip]
    def test_fermat(n, a=0):
        if not a:
            a = rd.randint(2, 100)

        if pow(a, n - 1, n) != 1:
            return False
        return True
\end{lstlisting}
\paragraph{c.} Probabilités empiriques d'erreurs sur  \verb|test_fermat| avec une base aléatoire pour des entiers inférieurs à $10^5$ sur des échantillons de tailles $10^5$ :\\
\begin{center}

\begin{tabular}{lll}
    Nombres de Carmichael :&0.9375\\
    Nombres Composés : & 0.00161\\
    Nombres aléatoires : &0.00132\\
\end{tabular}

\end{center}
\newpage
\section*{Exercice 4}
\paragraph{a.} \
\begin{lstlisting}[language=Python, belowskip=-1 \baselineskip]
def test_miller_rabin(n, k=3):
    h, m = 0, n - 1
    while m % 2 == 0:
        h += 1
        m //= 2
    for _ in range(k):
        a = random.randrange(2, n - 1)
        b = pow(a, m, n)
        if b == 1 or b == n - 1:
            continue
        for _ in range(h - 1):
            if b != n-1 and pow(b, 2, n) == 1:
                return False
            if b == n - 1:
                break
            b = pow(b, 2, n)
        if b != n - 1:
            return False
    return True
\end{lstlisting} 
\paragraph{c.} Probabilités empiriques d'erreurs sur  \verb|test_miller_rabin| avec une précision T= 3 pour des entiers inférieurs à $10^5$ sur des échantillons de tailles $10^7$ :\\
\begin{center}
    
    \begin{tabular}{lll}
        Nombres de Carmichael :&0.0008469\\
        Nombres Composés : & 0.0000029\\
        Nombres aléatoires : &0.0000018\\
    \end{tabular}
    
\end{center}
Le nombre d'erreur (18 pour 10 millions de tests en nombres aléatoires) n'est pas suffisamment significatif pour conclure quant à l'impact de la primalité des nombres testés sur l'algorithme ; on peut néanmoins remarquer qu'il est plus fiable que l'algorithme du test de Fermat de l'exercice précédent.
\paragraph{d.} \
\begin{lstlisting}[language=Python, belowskip=-1 \baselineskip]
def gen_rsa(t):
    res = []
    for n in range(2 ** (t - 1), 2 ** t):
        if test_miller_rabin(n):
            res.append(n)
            if len(res) == 2:
                break
    if len(res) < 2:
        raise ValueError("Intervalle trop court")
    return res[0] * res[1]
\end{lstlisting}
\end{document}
